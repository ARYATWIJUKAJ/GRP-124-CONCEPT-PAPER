\documentclass[14pt]{article}
\usepackage{graphicx}
\begin{document}


\title{CONSTRUCTION OF FINITE AUTOMATA FROM EXAMPLES.}

\author{BY ARYATWIJUKA JUDITH.REG.NO. 15/U/3718/PS. ST.NO.215004216 ,
             MASSANGANZIRA ISAAC N . 15/U/7440/EVE . 215014739}
\date{\today}

\section{Abstract}
The problem in this paper is guided learning of finite automata from examples. Given positive sample strings and negative sample stings, a finite automation can use to accept or reject a certain language and we want to show that many problems can be solved using any of the methods.
\section{Introduction}
The problem domain is restricted to be only over {0, 1}. Furthermore, since every none deterministic finite automaton, we deal only with deterministic finite automation, that is, there is at most 1-arrow and one 0-arrow from each state. Thus in this paper the terms “finite automation”, “automaton”, “machine” all mean deterministic finite automaton.
 Given a string, if there is a transition from the initial state to any of the final states, then the string is accepted by the automation, otherwise it is neglected.

\section{The problem}
Given a right list (a set of positive sample strings) and a wrong list (a set of a negative sample strings), we should be able to do the following tasks;
1)	To find automation with n states that accepts all strings in the right-list but none in the wrong-list.
2)	To find automation that accepts all strings in the right list but none in the wrong-list
3)	To find an automation with the fewest states (simplest machine) that accepts all strings in the right-list but none in the wrong-list.
The first task and the third task are complete problems. The first task is known as construction of finite automata, and the third as simplification of the finite automata.
The second task is trivial because one can easily construct a trivial machine that accepts exactly all the strings in the right-list but nothing else.
\section{Past work}
Feldman, Gips, Homing and Reder built a system that constructs a grammar in BNF from given examples. It takes only positive examples, and its problem domain is context-free languages.
You can try [Feldman 69] to know how their system worked there are sample runs.
The system first constructs a”trivial” grammar, and then simplifies it. Their system requires us to provide nicely-chosen examples, and it cannot solve from poorly organised examples.
Bierman and Feldman then built a system that constructs a finite automaton from given examples. Although it takes only positive examples, they showed an application to the case where both positive and negative examples are given. The algorithm also requires nicely-chosen examples and they showed the method to choose the examples from a regular set “nicely” so that it always turns out the simplest machine. However if the examples are not nicely-chosen, their system hardly turns out the simplest machine.

\section{Aims and Objectives}
\subsection{Aim or General Objective}
The main objective is to construct finite automata from examples that can be able to solve the tasks mentioned above regardless of the examples chosen.
\subsection{Specific Objectives}
 It is also to present the theory of finite automata, as the first step towards learning advanced topics such as compiler  
design. To also discuss the applications of finite automata towards text processing and to develop an understanding of computation through Turing Machines.
\section{Research Scope}
In this study, we cover the 7 sample problems, thatwe will consider throughout the paper.

\section{Methodology}
We first construct a random machine with 8 states. We next make a copy of this machine, where the copy is slightly altered from the original by an operator mutate. We compare the new machine with the original by evaluation function E. The better machine is called current generation and we make a copy of this machine , and so forth. The worse machine is simply discarded. The operator mutate and the evaluation function E  are defined more precisely in the following.
Operator mutate. Taking a machine ((A1, B1, F1)...(A8, B8,F8)) as its arguement, the operator mutate chooses one digit  randomly, and replaces it by another digit. That is, the mutation in our algorithm is randomly  one of the following; delete an arrow, insert an arrow , change the destination ofan arrow to another destination, make a non-final state and make a final state into a non-final state. 
\section{Conclusion}
Our new approach to construction of finite automata from given examples has been shown to work very nicely, despite the fact that the algorithm is quite simple. We saw that construction of finite automata with n states can be nicely done using hill climbing if n is a reasonable number.  
\end{document}